\documentclass[12pt,fleqn]{exam}
\usepackage{pifont}
\usepackage{dingbat}
\usepackage{amsmath,amssymb}
\usepackage{epsfig}
\usepackage[]{hyperref}
\usepackage{geometry}
\geometry{letterpaper, margin=0.5in}
\addpoints
\boxedpoints
\pointsinmargin
\pointname{pts}

\usepackage[activate={true,nocompatibility},final,tracking=true,kerning=true,factor=1100,stretch=10,shrink=10]{microtype}
\usepackage[american]{babel}
%\usepackage[T1]{fontenc}
\usepackage{fourier}
\usepackage{isomath}
\usepackage{upgreek,amsmath}
\usepackage{amssymb}

\newcommand{\dotprod}{\, {\scriptzcriptztyle
    \stackrel{\bullet}{{}}}\,}

\newcommand{\reals}{\mathbf{R}}
\newcommand{\lub}{\mathrm{lub}} 
\newcommand{\glb}{\mathrm{glb}} 
\newcommand{\complex}{\mathbf{C}}
\newcommand{\dom}{\mbox{dom}}
\newcommand{\cover}{{\mathcal C}}
\newcommand{\integers}{\mathbf{Z}}
\newcommand{\vi}{\, \mathbf{i}}
\newcommand{\vj}{\, \mathbf{j}}
\newcommand{\vk}{\, \mathbf{k}}
\newcommand{\bi}{\, \mathbf{i}}
\newcommand{\bj}{\, \mathbf{j}}
\newcommand{\bk}{\, \mathbf{k}}
\DeclareMathOperator{\Arg}{\mathrm{Arg}}
\DeclareMathOperator{\Ln}{\mathrm{Ln}}
\newcommand{\imag}{\, \mathrm{i}}

\usepackage{graphicx}
\newcommand\AM{{\sc am}}
\newcommand\PM{{\sc pm}}
     
\newcommand{\quiz}{1}
\newcommand{\term}{Spring}
\newcommand{\due}{Saturday 28 January at 11:59 \PM}
\begin{document}
\large
\vspace{0.1in}
\noindent\makebox[3.0truein][l]{{\bf MATH 250}}
{\bf Name:}  \\
\noindent \makebox[3.0truein][l]{\bf Homework   \quiz, \term \/ \the\year}
%{\bf Row:}\hrulefill\
\vspace{0.1in}

\begin{quote}
    \fbox{I have neither given nor received unauthorized assistance on this assignment.}
    \end{quote}
\noindent  Homework    \quiz\/  has questions 1 through  \numquestions \/ with 
a total of  \numpoints\/  points. For this assignment, \emph{neatly handwrite} 
your work on your own paper, digitize it, and upload it to Canvas.


\vspace{0.1in}

%\noindent{\textbf{Link to your Overleaf work: }}\url{XXX}

\begin{questions} 

\question [5] Define statements $A,B,C,D$, and $E$ as
\begin{itemize}
\item $A = $ The number $\uppi$ is rational.
\item $B = $ The integer  $4$ is even.
\item $C = $ We have $3 < \uppi$.
\item $D = $ The number $\sqrt{2}$ is not rational.
\item $E = $ The cosine function is an even function.
\end{itemize}

Using these statements, write an example of 
\begin{parts}

    \part  a conjunction of two statements that has a truth value of 
              false.
    \part  a conjunction of two statements that has a truth value of 
              true.         
    \part  a disjunction of two statements that has a truth value of 
              true.  
    \part  a disjunction of two statements that has a truth value of 
              false.  
    \part  a conditional statement with a truth value of true.
       
\end{parts}


\question[5] Explain why the sentence ``Larry likes lemon lollypops''
is not a proposition.

\question [5] Complete the truth table for $\lnot  (P \lor Q) \equiv (\lnot P) \land (\lnot  Q)$.

\vspace{0.1in}
\begin{tabular}{|c|c|c|c|c|c|c|c|}
\hline 
\(P\) & \(Q\) & \(\lnot P \) & \(\lnot Q \) & \(P \lor Q\) &  \( \lnot (P \lor Q) \)& \(\lnot P \land  \lnot  Q\) & \(\lnot  (P \lor Q) \equiv (\lnot P)\land (\lnot  Q) \) \\  \hline \hline
 T  & T & F & F & T & F & F & T   \\  \hline
 T  & F &   &   &   &   &   &  \\ \hline 
 F  & T &   &   &   &   &   &  \\ \hline 
 F  & F &   &   &   &   &   &  \\ \hline 
\end{tabular}



\begin{solution} 
\end{solution}




\end{questions}



\end{document}