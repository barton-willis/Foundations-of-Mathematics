\documentclass[12pt,fleqn,answers]{exam}
\usepackage{pifont}
\usepackage{dingbat}
\usepackage{amsmath,amssymb}
\usepackage{epsfig}
\usepackage[]{hyperref}
\usepackage{geometry}
\geometry{letterpaper, margin=0.5in}
\addpoints
\boxedpoints
\pointsinmargin
\pointname{pts}

\usepackage[activate={true,nocompatibility},final,tracking=true,kerning=true,factor=1100,stretch=10,shrink=10]{microtype}
\usepackage[american]{babel}
%\usepackage[T1]{fontenc}
\usepackage{fourier}
\usepackage{isomath}
\usepackage{upgreek,amsmath}
\usepackage{amssymb}

\newcommand{\dotprod}{\, {\scriptzcriptztyle
    \stackrel{\bullet}{{}}}\,}

\newcommand{\reals}{\mathbf{R}}
\newcommand{\lub}{\mathrm{lub}} 
\newcommand{\glb}{\mathrm{glb}} 
\newcommand{\complex}{\mathbf{C}}
\newcommand{\dom}{\mbox{dom}}
\newcommand{\cover}{{\mathcal C}}
\newcommand{\integers}{\mathbf{Z}}
\newcommand{\vi}{\, \mathbf{i}}
\newcommand{\vj}{\, \mathbf{j}}
\newcommand{\vk}{\, \mathbf{k}}
\newcommand{\bi}{\, \mathbf{i}}
\newcommand{\bj}{\, \mathbf{j}}
\newcommand{\bk}{\, \mathbf{k}}
\DeclareMathOperator{\Arg}{\mathrm{Arg}}
\DeclareMathOperator{\Ln}{\mathrm{Ln}}
\newcommand{\imag}{\, \mathrm{i}}

\usepackage{graphicx}
\newcommand\AM{{\sc am}}
\newcommand\PM{{\sc pm}}
     
\newcommand{\quiz}{2}
\newcommand{\term}{Spring}
\newcommand{\due}{Saturday 4 Febuary at 11:59 \PM}
\begin{document}
\large
\vspace{0.1in}
\noindent\makebox[3.0truein][l]{{\bf MATH 250}}
{\bf Name:}  \\
\noindent \makebox[3.0truein][l]{\bf Homework \quiz, \term \/ \the\year}
%{\bf Row:}\hrulefill\
\vspace{0.1in}

\begin{quote}
    \fbox{I have neither given nor received unauthorized assistance on this assignment.}
    \end{quote}
\noindent  Homework    \quiz\/  has questions 1 through  \numquestions \/ with 
a total of  \numpoints\/  points. For this assignment, use Overleaf to 
complete the assignment and upload the pdf to Canvas.


\vspace{0.1in}

%\noindent{\textbf{Link to your Overleaf work: }}\url{XXX}

\begin{questions} 

\question[5] Write the \emph{contrapositive} of the statement
\begin{quotation}
    If $x = 1$, then $x^2 = 1$.
\end{quotation}
\begin{solution}

\end{solution}

\question[5] Write a proof of the statement
\begin{quotation}
    If $x = 1$, then $x^2 = 1$.
\end{quotation}
\begin{solution}

\end{solution}

\question[5] Write the \emph{converse} of the statement
\begin{quotation}
    If $x = 1$, then $x^2 = 1$.
\end{quotation}
\begin{solution}

\end{solution}

\question[5]  Show that the statement
\begin{quotation}
    If $x^2 = 1$, then $x = 1$.
\end{quotation}
is false. To do this, find a specific number $x$ that makes the
statement false.
\begin{solution}

\end{solution}

\question Show that $(P \implies Q) \implies (Q \implies P)$ is not 
a tautology. To do this find one example of truth values for $P$ 
and $Q$ that make $(P \implies Q) \implies (Q \implies P)$ false.
\begin{solution}

\end{solution}

\question[5]  Show that $(P \equiv Q) \equiv \left ( (P \implies Q) 
\land (Q \implies P) \right )$ is a tautology. Do this by completing
the truth table

\begin{solution}

    \begin{tabular}{|c|c|c|c|c|c|}
        \hline
        $P$ & $Q$ & $P \equiv Q$ & $P \implies Q$ & $Q \implies P$ & $(P \implies Q) 
        \land (Q \implies P)$ \\ \hline \hline
        T & T &  &  & & \\ \hline
        T & F &  &  & & \\ \hline 
        F & T &  &  & & \\ \hline 
        F & F &  &  & & \\ \hline        
    \end{tabular}

\end{solution}
\end{questions}



\end{document}