\documentclass[12pt,fleqn,answers]{exam}
\usepackage{pifont}
\usepackage{dingbat}
\usepackage{amsmath,amssymb}
\usepackage{epsfig}
\usepackage[]{hyperref}
\usepackage{geometry}
\geometry{letterpaper, margin=0.75in}
\addpoints
\boxedpoints
\pointsinmargin
\pointname{pts}

\usepackage[activate={true,nocompatibility},final,tracking=true,kerning=true,factor=1100,stretch=10,shrink=10]{microtype}
\usepackage[american]{babel}
%\usepackage[T1]{fontenc}
\usepackage{fourier}
\usepackage{isomath}
\usepackage{upgreek,amsmath}
\usepackage{amssymb}

\newcommand{\dotprod}{\, {\scriptzcriptztyle
    \stackrel{\bullet}{{}}}\,}

\newcommand{\reals}{\mathbf{R}}
\newcommand{\lub}{\mathrm{lub}} 
\newcommand{\glb}{\mathrm{glb}} 
\newcommand{\complex}{\mathbf{C}}
\newcommand{\dom}{\mbox{dom}}
\newcommand{\cover}{{\mathcal C}}
\newcommand{\integers}{\mathbf{Z}}
\newcommand{\vi}{\, \mathbf{i}}
\newcommand{\vj}{\, \mathbf{j}}
\newcommand{\vk}{\, \mathbf{k}}
\newcommand{\bi}{\, \mathbf{i}}
\newcommand{\bj}{\, \mathbf{j}}
\newcommand{\bk}{\, \mathbf{k}}
\DeclareMathOperator{\Arg}{\mathrm{Arg}}
\DeclareMathOperator{\Ln}{\mathrm{Ln}}
\newcommand{\imag}{\, \mathrm{i}}

%\DeclareMathOperator{\dom}{\mathrm{dom}}
\DeclareMathOperator{\range}{\mathrm{range}}
\usepackage{graphicx}
\newcommand\AM{{\sc am}}
\newcommand\PM{{\sc pm}}
     
\newcommand{\quiz}{10}
\newcommand{\term}{Spring}
\newcommand{\due}{Saturday 15 April at 11:59 \PM}
\begin{document}
\large
\vspace{0.1in}
\noindent\makebox[3.0truein][l]{{\bf MATH 250}}
{\bf Name:}  \\
\noindent \makebox[3.0truein][l]{\bf Homework \quiz, \term \/ \the\year}
%{\bf Row:}\hrulefill\
\vspace{0.1in}

\begin{quote}
    “Study without desire spoils the memory, and it retains nothing that it takes in.”
    \hfill{\sc Leonardo da Vinci}
\end{quote}
\begin{quote}
      \fbox{I have neither given nor received unauthorized assistance on this assignment.}
    \end{quote}

\noindent  Homework    \quiz\/  has questions 1 through  \numquestions \/ with 
a total of  \numpoints\/  points. For this assignment, use Overleaf to typeset your work
and upload a pdf to Canvas. This assignment is due  \due.


\vspace{0.1in}

%\noindent{\textbf{Link to your Overleaf work: }}\url{XXX}

\begin{questions} 


    \question A four-digit integer has the form $d_1 d_2 d_3 d_4$,
    where $d_1, d_2 \dots, d_4 \in \{0,1,2,3,4,5,6,7,8,9\}$ and $d_1 \neq 0$.  
    Thus $1225$ and $9876$ are four-digit numbers, but $0123$ isn't a four-digit number 
    (it is a three digit number).
    
    In each of the following cases, determine the number of four-digit integers 
    that satisfy the given condition. Clearly explain your work with a few sentences.

    \begin{parts}
      \part [2] No additional restrictions on the digits.
 
      \begin{solution}

      \end{solution}
        
 
      \part [2] Every digit is even.
      \begin{solution}

      \end{solution}

      \part [2] There are no repeated digits. Thus $1078$ is admissible, but $1040$
      is not admissible.
      \begin{solution}

      \end{solution}

      \part [2] No two consecutive digits are equal. Thus $1313$ is admissible, but $1331$
      is not admissible.
      \begin{solution}

      \end{solution}

    \end{parts}

     \question [2] From a litter of six border collies named Patsy, George, Buddy, Larry,
     Joey, and Clover, I'm going to adopt three of them. How many distinct sets of
     three puppies can I adopt?  Clearly explain your work with a few sentences.

     \begin{solution}

     \end{solution}
     \question [2] The prime factorization of $220870604912736225056311099$ is
     ${{19}^{3}} \times  {{37}^{4}} \times {{107}^{8}}$. Every divisor of $220870604912736225056311099$
     has the form $19^\ell \times 37^m \times 107^n$, where $\ell \in \{0,1,2,3\}$,
     $m \in \{0,1,2,3,4\}$, and $n \in \{0,1,2,3,4,5,6,7,8\}$. How many divisors of 
     $220870604912736225056311099$ are there?

     \begin{solution}

     \end{solution}

\end{questions}

\end{document}