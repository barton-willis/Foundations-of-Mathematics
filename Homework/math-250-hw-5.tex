\documentclass[12pt,fleqn,answers]{exam}
\usepackage{pifont}
\usepackage{dingbat}
\usepackage{amsmath,amssymb}
\usepackage{epsfig}
\usepackage[]{hyperref}
\usepackage{geometry}
\geometry{letterpaper, margin=0.75in}
\addpoints
\boxedpoints
\pointsinmargin
\pointname{pts}

\usepackage[activate={true,nocompatibility},final,tracking=true,kerning=true,factor=1100,stretch=10,shrink=10]{microtype}
\usepackage[american]{babel}
%\usepackage[T1]{fontenc}
\usepackage{fourier}
\usepackage{isomath}
\usepackage{upgreek,amsmath}
\usepackage{amssymb}

\newcommand{\dotprod}{\, {\scriptzcriptztyle
    \stackrel{\bullet}{{}}}\,}

\newcommand{\reals}{\mathbf{R}}
\newcommand{\lub}{\mathrm{lub}} 
\newcommand{\glb}{\mathrm{glb}} 
\newcommand{\complex}{\mathbf{C}}
\newcommand{\dom}{\mbox{dom}}
\newcommand{\cover}{{\mathcal C}}
\newcommand{\integers}{\mathbf{Z}}
\newcommand{\vi}{\, \mathbf{i}}
\newcommand{\vj}{\, \mathbf{j}}
\newcommand{\vk}{\, \mathbf{k}}
\newcommand{\bi}{\, \mathbf{i}}
\newcommand{\bj}{\, \mathbf{j}}
\newcommand{\bk}{\, \mathbf{k}}
\DeclareMathOperator{\Arg}{\mathrm{Arg}}
\DeclareMathOperator{\Ln}{\mathrm{Ln}}
\newcommand{\imag}{\, \mathrm{i}}

\usepackage{graphicx}
\newcommand\AM{{\sc am}}
\newcommand\PM{{\sc pm}}
     
\newcommand{\quiz}{5}
\newcommand{\term}{Spring}
\newcommand{\due}{Saturday 11 March at 11:59 \PM}
\begin{document}
\large
\vspace{0.1in}
\noindent\makebox[3.0truein][l]{{\bf MATH 250}}
{\bf Name:}  \\
\noindent \makebox[3.0truein][l]{\bf Homework \quiz, \term \/ \the\year}
%{\bf Row:}\hrulefill\
\vspace{0.1in}

\begin{quote}
    “To learn, one must be humble. But life is the great teacher.”
    \hfill{\sc James Joyce}
\end{quote}
\begin{quote}
      \fbox{I have neither given nor received unauthorized assistance on this assignment.}
    \end{quote}

\noindent  Homework    \quiz\/  has questions 1 through  \numquestions \/ with 
a total of  \numpoints\/  points. For this assignment, \emph{neatly handwrite} 
your work on your own paper, digitize it, and upload it to Canvas.
This assignment is due  \due.


\vspace{0.1in}

%\noindent{\textbf{Link to your Overleaf work: }}\url{XXX}

\begin{questions} 

\question [1] Show that $\{(2,3)\}$ is a transitive relation on $\integers$.


\question [1] Show that $\{(1,3), (3,4), (1,4)\}$ is a transitive relation on $\integers$.

\question [1] Give an example of a set $A$ and transitive relations
$R$ and $R^\prime$ on $A$ such that $R \cup R^\prime$ is not 
a transitive relation on $A$. (Hint: An example is hiding in plain sight.)

\question On the set $\reals$, define a relation $\{(a,b) \in \reals \times \reals \mid a \neq b \}$.

\begin{parts} 
    
    \part [1] Is this relation \emph{reflexive}? If so, prove it;
    if not, give an example that shows that it is not reflexive.

    \part [1] Is this relation \emph{symmetric}? If so, prove it;
        if not, give an example that shows that it is not symmetric.

    \part [1] Is this relation \emph{transitive}? If so, prove it;
        if not, give an example that shows that it is not transitive.

\end{parts}

\question [1] To say that a function $f$ is constant means
\(
     \left(\exists \,\,  C  \right)
     \left(\forall \,\, x \in \dom(f) \right)
     \left(f(x) = C \right)
\). Let $F$ be the set of all real valued functions from $\reals$ to 
$\reals$. Define a relation $E$ on $F$ as 
\begin{equation*}
    E = \{(f,g) \mid f - g \mbox{ is a constant function} \}.
\end{equation*}
If $E$ is an equivalence relation on $F$, prove it; if not, 
give an example that shows that it is not an equivalence relation.



\end{questions}






\end{document}