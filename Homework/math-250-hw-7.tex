\documentclass[12pt,fleqn,answers]{exam}
\usepackage{pifont}
\usepackage{dingbat}
\usepackage{amsmath,amssymb}
\usepackage{epsfig}
\usepackage[]{hyperref}
\usepackage{geometry}
\geometry{letterpaper, margin=0.75in}
\addpoints
\boxedpoints
\pointsinmargin
\pointname{pts}

\usepackage[activate={true,nocompatibility},final,tracking=true,kerning=true,factor=1100,stretch=10,shrink=10]{microtype}
\usepackage[american]{babel}
%\usepackage[T1]{fontenc}
\usepackage{fourier}
\usepackage{isomath}
\usepackage{upgreek,amsmath}
\usepackage{amssymb}

\newcommand{\dotprod}{\, {\scriptzcriptztyle
    \stackrel{\bullet}{{}}}\,}

\newcommand{\reals}{\mathbf{R}}
\newcommand{\lub}{\mathrm{lub}} 
\newcommand{\glb}{\mathrm{glb}} 
\newcommand{\complex}{\mathbf{C}}
\newcommand{\dom}{\mbox{dom}}
\newcommand{\cover}{{\mathcal C}}
\newcommand{\integers}{\mathbf{Z}}
\newcommand{\vi}{\, \mathbf{i}}
\newcommand{\vj}{\, \mathbf{j}}
\newcommand{\vk}{\, \mathbf{k}}
\newcommand{\bi}{\, \mathbf{i}}
\newcommand{\bj}{\, \mathbf{j}}
\newcommand{\bk}{\, \mathbf{k}}
\DeclareMathOperator{\Arg}{\mathrm{Arg}}
\DeclareMathOperator{\Ln}{\mathrm{Ln}}
\newcommand{\imag}{\, \mathrm{i}}

%\DeclareMathOperator{\dom}{\mathrm{dom}}
\DeclareMathOperator{\range}{\mathrm{range}}
\usepackage{graphicx}
\newcommand\AM{{\sc am}}
\newcommand\PM{{\sc pm}}
     
\newcommand{\quiz}{7}
\newcommand{\term}{Spring}
\newcommand{\due}{Saturday 1 April at 11:59 \PM}
\begin{document}
\large
\vspace{0.1in}
\noindent\makebox[3.0truein][l]{{\bf MATH 250}}
{\bf Name:}  \\
\noindent \makebox[3.0truein][l]{\bf Homework \quiz, \term \/ \the\year}
%{\bf Row:}\hrulefill\
\vspace{0.1in}

\begin{quote}
    “To learn, one must be humble. But life is the great teacher.”
    \hfill{\sc James Joyce}
\end{quote}
\begin{quote}
      \fbox{I have neither given nor received unauthorized assistance on this assignment.}
    \end{quote}

\noindent  Homework    \quiz\/  has questions 1 through  \numquestions \/ with 
a total of  \numpoints\/  points. For this assignment, \emph{neatly hand write} 
your work on your own paper, digitize it, and upload it to Canvas.
This assignment is due  \due.


\vspace{0.1in}

%\noindent{\textbf{Link to your Overleaf work: }}\url{XXX}

\begin{questions} 

\question Let $A$ and $B$ be subsets of $\reals$. Any function
from $A$ to $B$ is called a \emph{real valued function of one real variable}.
For any two such functions $F$ and $G$, define
\begin{align*}
&F+G = x \in \dom(F) \cap \dom(G) \mapsto F(x) + G(x),\\
&F G = x \in \dom(F) \cap \dom(G) \mapsto F(x)  G(x),\\
&\frac{F}{G} = x \in \dom(F) \cap \dom(G) \setminus 
\{x \in \reals | G(x) = 0 \} \mapsto \frac{F(x)}{G(x)}.
\end{align*}

\begin{parts}
\part[10] Find an example of two \emph{real valued functions of one real variable} 
$F$ and $G$ such that both $F$ and $G$ are one-to-one, but $F+G$ is not 
one-to-one. Justify your example.

\part[10] Find an example of two \emph{real valued functions of one real variable} $F$ and $G$ such that both $F$ and $G$ are one-to-one, but $F G$ is not 
one-to-one. Justify your example.

\part[10] Define $F = x \in \reals_{\neq 0} \mapsto x + \frac{1}{x}$. Find
\(\dom(1/F)\).

\end{parts}



\question On the set $\reals$ define the equivalence relation $(x \sim y) \equiv
\sin(x) = \sin(y)$. Give an explicit representation for the equivalence class $[1/2]$.
(Okay--the set $[1/2]$ has infinitely many members, so no, you can't list all the
members, but you can express the set $[1/2]$ in set builder notation.)


\question Define $\Phi = k \in \integers \mapsto \begin{cases} 2k & k \leq 0 \\
    2 k + 1 & k > 0 \end{cases}$.


\begin{parts}

\part [10] Is $\Phi$ one-to-one?  If so, prove it; if not, prove that too.

\part [10] Find  $\range(\Phi)$.

\end{parts}

\question The prime factorization of an integer is unique. On consequence of this is that
if $2^k \times 3^\ell = 2^{k^\prime} \times 3^{\ell^\prime}$, where $k,\ell, k$, and
$k^\prime$ are non negative integers, then $k=k^\prime$ and $\ell = \ell^\prime$.
Define a function \(\Psi = (k,n) \in \integers_{\geq 0} \times \integers_{\geq 0} 
\mapsto 2^k 3^n$.

\begin{parts}

    \part [10] Show that the function $\Psi$ is one-to-one.

    \part [10] Show that $\range(\Psi) \neq \integers_{\geq 0}$
\end{parts}

\end{questions}

\end{document}