\documentclass[12pt,fleqn,answers]{exam}
\usepackage{pifont}
\usepackage{dingbat}
\usepackage{amsmath,amssymb}
\usepackage{geometry}
\geometry{letterpaper, margin=0.5in}
\addpoints
\boxedpoints
\pointsinmargin
\pointname{pts}

\usepackage[final]{microtype}
\usepackage[american]{babel}
\usepackage{fourier}
\usepackage{isomath}
\usepackage{upgreek}
\usepackage{amsmath}
\usepackage{amssymb}

\newcommand{\reals}{\mathbf{R}}
\newcommand{\integers}{\mathbf{Z}}
\DeclareMathOperator{\dom}{\mathrm{dom}}
\DeclareMathOperator{\range}{\mathrm{range}}

     
\newcommand{\quiz}{8}
\newcommand{\term}{Spring}
\newcommand{\due}{Saturday 8 April at 11:59 pm}
\begin{document}
\large
\vspace{0.1in}
\noindent\makebox[3.0truein][l]{ \textbf{MATH 250}}
{\bf Name:}  \\
\noindent \makebox[3.0truein][l]{\textbf{Homework \quiz, \term \/ \the\year}}
%{\bf Row:}\hrulefill\
\vspace{0.1in}

\begin{quote}
   % “To learn, one must be humble. But life is the great teacher.”
   % \hfill{\sc James Joyce}
    The more I read, the more I acquire, the more certain I am that 
    I know \mbox{nothing.} \hfill{\sc François-Marie Arouet}
\end{quote}
\begin{quote}
      \fbox{I have neither given nor received unauthorized assistance on this assignment.}
    \end{quote}

\noindent  Homework    \quiz\/  has questions 1 through  \numquestions \/ with 
a total of  \numpoints\/  points. For this assignment, \textbf{use Overleaf 
to typeset your work and upload a pdf to Canvas.} The assignment is due \due.
\vspace{0.1in}

%\noindent{\textbf{Link to your Overleaf work: }}\url{XXX}

\begin{questions} 

\question [10] Let $A,B$, and $C$ be countable sets. Show that $A \times B \times C$
is countable.  Remember that  $A \times B \times C$ is the set of three-tuples; specifically
\begin{equation*}
    A \times B \times C = \{(a,b,c) \,\,| \,\, a \in A, b \in B, c \in C \}.
\end{equation*}
\textbf{Hint} You need to find $\Phi \in A \times B \times C \to \integers_{\geq 0}$
that is one-to-one. But you do \emph{not} need to make $\Phi$ onto. 

\question [10]  Let $A$ and $B$ be nonempty finite sets. Show that $A \setminus B$ 
is finite. (The sets needn't be nonempty, but assuming they are both
nonempty eliminates two special cases.)
\begin{solution}

\end{solution}
\question [10] Let $A$ and $B$ be nonempty finite sets. Show that $A \cup B$ 
is finite. (Again, the sets needn't be nonempty.)
\begin{solution}

\end{solution}
    

\textbf{Hints} There is $M \in \integers_{\geq 1}$ and a 
bijection $\Phi \in A \to 1 \dots M$. Similarly, there is 
$N \in \integers_{\geq 1}$ and a bijection $\Psi \in B \to 1 \dots N$.
For some integer $L$, you need to define a bijection from $A \cup B$ 
to $ 1 \dots L$. If $A$ and $B$ were disjoint, you 
could define 
\begin{equation*}
  \Omega = x \in A \cup B \mapsto 
   \begin{cases} 
      \Phi(x) & x \in A \\
      \Psi(x) + M & x \in B
   \end{cases}.
\end{equation*}
Then $\Omega$ is a bijection from $A \cup B$ to $1 \dots M + N$.
But this doesn't work if $A \cap B \neq \varnothing$. Actually
when $A$ and $B$ aren't disjoint, $ \Omega$ isn't a function--if
$w \in A \cap B$, what's the value of $\Omega(w)$? 
But the definition of $\Omega$ can be tweaked to make 
$\Omega$ a bijection.

\begin{solution}

\end{solution}

\end{questions}

\end{document}