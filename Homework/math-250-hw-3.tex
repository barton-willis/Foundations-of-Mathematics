\documentclass[12pt,fleqn,answers]{exam}
\usepackage{pifont}
\usepackage{dingbat}
\usepackage{amsmath,amssymb}
\usepackage{epsfig}
\usepackage[]{hyperref}
\usepackage{geometry}
\geometry{letterpaper, margin=0.5in}
\addpoints
\boxedpoints
\pointsinmargin
\pointname{pts}

\usepackage[activate={true,nocompatibility},final,tracking=true,kerning=true,factor=1100,stretch=10,shrink=10]{microtype}
\usepackage[american]{babel}
%\usepackage[T1]{fontenc}
\usepackage{fourier}
\usepackage{isomath}
\usepackage{upgreek,amsmath}
\usepackage{amssymb}

\newcommand{\dotprod}{\, {\scriptzcriptztyle
    \stackrel{\bullet}{{}}}\,}

\newcommand{\reals}{\mathbf{R}}
\newcommand{\lub}{\mathrm{lub}} 
\newcommand{\glb}{\mathrm{glb}} 
\newcommand{\complex}{\mathbf{C}}
\newcommand{\dom}{\mbox{dom}}
\newcommand{\cover}{{\mathcal C}}
\newcommand{\integers}{\mathbf{Z}}
\newcommand{\vi}{\, \mathbf{i}}
\newcommand{\vj}{\, \mathbf{j}}
\newcommand{\vk}{\, \mathbf{k}}
\newcommand{\bi}{\, \mathbf{i}}
\newcommand{\bj}{\, \mathbf{j}}
\newcommand{\bk}{\, \mathbf{k}}
\DeclareMathOperator{\Arg}{\mathrm{Arg}}
\DeclareMathOperator{\Ln}{\mathrm{Ln}}
\newcommand{\imag}{\, \mathrm{i}}

\usepackage{graphicx}
\newcommand\AM{{\sc am}}
\newcommand\PM{{\sc pm}}
     
\newcommand{\quiz}{3}
\newcommand{\term}{Spring}
\newcommand{\due}{Saturday 11 Febuary at 11:59 \PM}
\begin{document}
\large
\vspace{0.1in}
\noindent\makebox[3.0truein][l]{{\bf MATH 250}}
{\bf Name:}  \\
\noindent \makebox[3.0truein][l]{\bf Homework \quiz, \term \/ \the\year}
%{\bf Row:}\hrulefill\
\vspace{0.1in}

\begin{quote}
 ``Mathematics is common sense.'' \hfill \sc{Errett Bishop} (1928--1983)
\end{quote}
Errett Bishop was born in Newton, Kansas and was the son of a Wichita State
mathematics professor. He is best known for his book \emph{Foundations of Constructive Analysis}.
Constructive mathematics rejects the Law of excluded middle.

\begin{quote}
    \fbox{I have neither given nor received unauthorized assistance on this assignment.}
    \end{quote}
\noindent  Homework    \quiz\/  has questions 1 through  \numquestions \/ with 
a total of  \numpoints\/  points. For this assignment, use Overleaf to 
complete the assignment and upload the pdf to Canvas.


\vspace{0.1in}

%\noindent{\textbf{Link to your Overleaf work: }}\url{XXX}

\begin{questions} 

\question [5] Show that for all positive integers $p$ and $q$ that
$3 q^2 \neq p^2$.  

To do this, use the fact that every positive integer can be
uniquely expressed in the form $3^e n$, where $e$ is a nonnegative
integer and $n$ is not divisible by 3. For example, $24 = 3^1 \times 8$
(and 8 is not divisible by 3); and $963 = 3^2 \times 107$ 
(and 107 is not divisible by 3). Uniqueness of this representation
means that
\begin{equation*}
  \left[3^e n = 3^{e^\prime} n^\prime \right] \equiv
  \left[(e = e^\prime) \land (n = n^\prime) \right].
\end{equation*}
Equivalently
\begin{equation*}
    \left[3^e n \neq  3^{e^\prime} n^\prime \right] \equiv
    \left[(e \neq e^\prime) \lor (n \neq n^\prime) \right].
  \end{equation*}

\begin{solution}

\end{solution}

\question [5] For all positive real numbers $x$ and $y$ with $x \neq y$, show that
\begin{equation*}
   \frac{x}{y} + \frac{y}{x} > 2.
\end{equation*}
To do this, assume that $\frac{x}{y} + \frac{y}{x} \leq   2.$ Use some
algebra facts to derive a contradiction.
\end{questions}



\end{document}