\documentclass[12pt,fleqn]{exam}
\usepackage{pifont,bbding,url}
\usepackage{dingbat}
\usepackage{amsmath}
\usepackage{centernot}
\usepackage{fleqn}
\usepackage{epsfig}
\usepackage{pdfpages}
\usepackage{fourier} %{mathptm}
\usepackage[activate={true,nocompatibility},final,tracking=true,kerning=true,spacing=true,factor=1100,stretch=10,shrink=10]{microtype}
\usepackage{xcolor}
\usepackage{enumerate}

\usepackage{mathtools}
\DeclarePairedDelimiter\ceil{\lceil}{\rceil}
\DeclarePairedDelimiter\floor{\lfloor}{\rfloor}
\DeclarePairedDelimiter{\parens}{\lparen}{\rparen}
\newcommand{\dom}{\mathrm{dom}}
\newcommand{\range}{\mathrm{range}}
\usepackage{amssymb, enumerate}
\shadedsolutions
%\definecolor{SolutionColor}{rgb}{0.9,1,1}
\definecolor{SolutionColor}{rgb}{1,1,0.7}
\addpoints
\boxedpoints
\pointsinmargin
\pointname{pts}

\newcommand{\dotprod}{\, {\scriptzcriptztyle \stackrel{\bullet}{{}}}\,}
\begin{document}

\newcommand{\reals}{\mathbf{R}}
\newcommand{\integers}{\mathbf{Z}}
\newcommand{\bi}{\mathbf{i}}
\newcommand{\bj}{\mathbf{j}}
\newcommand{\bk}{mathbf{k}}

\newcommand{\Mod}[1]{\ \mathrm{mod}\ #1}
\newcommand{\card}{\ \mathrm{card}}

\newcommand{\ex}{III}
\newenvironment{alphalist}{
  \begin{enumerate}[(a)]
    \addtolength{\itemsep}{-1.0\itemsep}}
  {\end{enumerate}}

\newenvironment{handlist}{
  \begin{enumerate}[\leftthumbsup]
    \addtolength{\itemsep}{-1.0\itemsep}}
  {\end{enumerate}}

\large
\vspace{0.1in}
\noindent\makebox[3.0truein][l]{{\bf MATH 250}}
{\bf Name:}\hrulefill\
\noindent \makebox[3.0truein][l]{\bf Review for Exam \ex}
{\bf Row:}\hrulefill\

\large

\vspace{0.1in}

\noindent Exam \ex\/  has questions 1 through  \numquestions \/ with a total of \numpoints \/ points.  This exam is printed on both sides of the paper.

\begin{questions}

\question From a litter of eight border collies, how many subsets of puppies with cardinality four are there?  Explain.

 \question A five digit integer has the form $d_1 d_2 d_3 d_4 d5$,
    where $d_1, d_2 \dots, d_5 \in \{0,1,2,3,4,5,6,7,8,9\}$ and $d_1 \neq 0$.  
      In each of the following cases, determine the number of five digit integers 
    that satisfy the given condition. 
    \textbf{Clearly explain your work with a few sentences.}

    \begin{parts}
      \part [2] No additional restrictions on the digits.
 
      \begin{solution}



      \end{solution}
      
      
        
       \part [2] There are no repeated digits. 
      \begin{solution}

      \end{solution}

\end{parts}

\question [10] Let $A,B$ and $C$ be finite sets. Given that $\card(A) = 46, \card(B) = 107$, and $\card(A \cap B) = 12$, find
$\card(A \cup B)$.
\begin{solution}%[3.0in]
\end{solution}

\newpage

\question [10] Find a bijection from $[0,1]$  to $[-1,1]$.
\begin{solution}%[3.0in]
\end{solution}

\newpage

\question [10] Show that the set $\integers$ is countable. To do this, you must construct a one-to-one function from $\integers$ to $\integers_{\geq 0}$.

\begin{solution}%[3.0in]
\end{solution}

\newpage



\newpage

\question [10]  A recursive definition of a sequence $F$ is
\(
  F_n = \begin{cases} 1  & \mbox{if } n = 0 \\ \frac{1}{4} F_{n-1} + \frac{3n+1}{4}  &\mbox{if } n \in \integers_{\geq 1} \end{cases}.
\)
Use induction to show that
\(
   \left(\forall n \in \integers_{\geq 0}\right) \left(F_n =  n + \frac{1}{4^n}  \right).
\)

\newpage

\question [10]  Let $A$ and $B$ be countable sets. Show that $A \cup B$ is countable.
\end{questions}
%\includepdf[pages={1-}]{cheat_sheet.pdf}     
\end{document}