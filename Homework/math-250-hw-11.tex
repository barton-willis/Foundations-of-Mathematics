\documentclass[12pt,fleqn,answers]{exam}
\usepackage{pifont}
\usepackage{dingbat}
\usepackage{amsmath,amssymb}
\usepackage{epsfig}
\usepackage[]{hyperref}
\usepackage{geometry}
\geometry{letterpaper, margin=0.75in}
\addpoints
\boxedpoints
\pointsinmargin
\pointname{pts}

\usepackage[activate={true,nocompatibility},final,tracking=true,kerning=true,factor=1100,stretch=10,shrink=10]{microtype}
\usepackage[american]{babel}
%\usepackage[T1]{fontenc}
\usepackage{fourier}
\usepackage{isomath}
\usepackage{upgreek,amsmath}
\usepackage{amssymb}
\usepackage[super]{nth}

\newcommand{\dotprod}{\, {\scriptzcriptztyle
    \stackrel{\bullet}{{}}}\,}

\newcommand{\reals}{\mathbf{R}}
\newcommand{\lub}{\mathrm{lub}} 
\newcommand{\glb}{\mathrm{glb}} 
\newcommand{\complex}{\mathbf{C}}
\newcommand{\dom}{\mbox{dom}}
\newcommand{\cover}{{\mathcal C}}
\newcommand{\integers}{\mathbf{Z}}
\newcommand{\vi}{\, \mathbf{i}}
\newcommand{\vj}{\, \mathbf{j}}
\newcommand{\vk}{\, \mathbf{k}}
\newcommand{\bi}{\, \mathbf{i}}
\newcommand{\bj}{\, \mathbf{j}}
\newcommand{\bk}{\, \mathbf{k}}
\DeclareMathOperator{\Arg}{\mathrm{Arg}}
\DeclareMathOperator{\Ln}{\mathrm{Ln}}
\newcommand{\imag}{\, \mathrm{i}}

%\DeclareMathOperator{\dom}{\mathrm{dom}}
\DeclareMathOperator{\range}{\mathrm{range}}
\usepackage{graphicx}
\newcommand\AM{{\sc am}}
\newcommand\PM{{\sc pm}}
     
\newcommand{\quiz}{9}
\newcommand{\term}{Spring}
\newcommand{\due}{Saturday 29 April at 11:59 \PM}
\begin{document}
\large
\vspace{0.1in}
\noindent\makebox[3.0truein][l]{{\bf MATH 250}}
{\bf Name:}  \\
\noindent \makebox[3.0truein][l]{\bf Homework, \term \/ \the\year}
%{\bf Row:}\hrulefill\
\vspace{0.1in}

\begin{flushleft}
  \emph{
    “Money buys everything except love, personality, freedom, immortality, silence, peace.”}
    \hfill \sc{Carl Sandburg}
 \end{flushleft}

%\begin{quote}
  %    \fbox{I have neither given nor received unauthorized assistance on this assignment.}
  %  \end{quote}

\noindent  This homework   /  has questions 1 through  \numquestions \/ with 
a total of  \numpoints\/  points. For this assignment, neatly hand write your solutions, digitize your work, and turn it into Canvas. It is \due.

By the way:  This homework is based on a homework problem from the book \emph{Abstract Algebra}, by I.N. Herstein that I was
assigned as an undergraduate.  This book is still in print---the \nth{3} edition can be purchased from the retailer Target with a suggested age of  22 years and up.

 When I worked this homework assignment, I marveled over the  function $\otimes$ and thought it was quite mysterious and \emph{pined} to  know how such a crazy example  could be invented. But I was too pusillanimous to ask my professor Dr.\  Dressler\footnote{He  is possibly best known for 
 the conjecture that between any two positive integers having the same prime factors there is a prime.  For example, there must be 
 a prime between $5 \times 7 \times 107$ and $5^2\times 7 \times 107$.  Actually there are lots of them, for example 3761 is prime
 and $5 \times 7 \times 107 < 3761 < 5^2\times 7 \times 107$.  I'm not sure if his conjecture has been resolved.
 
 I'm pretty sure that Dr.\ Dressler didn't have
 much appreciation for applications of mathematics. He once told me that the problems in differential equations could be mostly eliminated
 by simply giving names to more functions.  What's the solution to the DE $y^{\prime\prime} = x^5 + x^3$.  No problem, the solution
 is $y= \mbox{Larry}(x)$; what's the solution to the DE $y^{\prime\prime} =  x + \cos(x)$.  No problem, the solution
 is $y= \mbox{George}(x)$.  In one of his books, Nobel Laureate Richard Feynman explains how his father taught him the 
  difference between \emph{understanding} something  and \emph{naming} it.  Maybe Dr.\  Dressler disagreed with Richard Feynman, or maybe he wasn't serious, I don't know. }
for an explanation on how might one invent such a crazy example.

I'm not sure when I learned the trick behind this function, but it is based on the associativity of function composition. Knowing 
the trick allows for constructing other examples of an associative and commutative function defined on pairs of n-tuples.  The particular
function in this problem defines what is known as hyperbolic numbers or perplex numbers.\footnote{The 1972 publication HAKMEM, also known as AI Memo 239, 
has a description of the trick that can be used to invent this example.}

\vspace{0.1in}

%\noindent{\textbf{Link to your Overleaf work: }}\url{XXX}

\begin{questions} 

\question Define a function $\otimes$ by
\begin{equation*}
   \otimes = \left  ((a,b), (c,d) \right) \in \reals^2 \times \reals^2  \mapsto
      (ac + bd, ad + bc) \in \reals^2.
\end{equation*}
The input to  $\otimes$  is an ordered pair of ordered pairs and the output is an ordered pair. Yikes!  Maybe an example
will help clarify.  Using infix notation, we have, for example,
\begin{align*}
  (1,2) \otimes (3,4) &= (1 \times 3 + 2 \times 4, 1 \times 4 + 2 \times 3)
                       = (11,10). \\
  (1,2) \otimes (1,0) &= (1 \times 1 + 2 \times 0, 1 \times 0 + 2 \times 1)
                       = (1,2).                      
\end{align*}

\begin{parts}

    \part [5] Show that $\otimes$ is commutative. That is, show that
    \begin{equation*}
        \left(\forall a,b,c,d \in \reals\right)
         \left((a,b) \otimes (c,d) = (c,d) \otimes (a,b) \right).
    \end{equation*}

     \part [5] Show that $\otimes$ is associative. That is, show that
     \begin{equation*}
        \left(\forall a,b,c,d,e,f \in \reals\right)
         \left((a,b) \otimes ((c,d) \otimes (e,f)) = 
         ((a,b) \otimes (c,d)) \otimes (e,f)  \right).
     \end{equation*}

    \part [5] Find $(x,y) \in \reals^2$ such that
    $
       \left(\forall a,b \in \reals\right)\left((a,b) \otimes (x,y)
         = (a,b)\right).
    $
    You have found the (and I mean the) identity element for $\otimes$.

    \part [5] Show that the solution set to 
    $(1,-1) \otimes (x,y) = (1,0)$ is empty.


    
    \part [5] Does the set $\reals^2$ along with the function $\otimes$ form a group?  Explain. 
\end{parts}
    

\end{questions}

\end{document}