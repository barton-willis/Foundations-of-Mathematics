\documentclass[12pt]{article}
%\documentclass[border=0.1cm]{standalone}
\usepackage{wasysym}
\usepackage{phonenumbers}
\usepackage{marvosym }
\usepackage{xcolor}
\usepackage[super]{nth}
\usepackage[paperwidth=8.5in,paperheight=11in,margin=0.45in]{geometry} 
\usepackage[UKenglish]{babel}
\usepackage[UKenglish]{isodate}% http://ctan.org/pkg/isodate
\usepackage{hyperref}
\usepackage[UKenglish]{babel}
\usepackage[UKenglish]{isodate}% http://ctan.org/pkg/isodate
\usepackage{hyperref}
\hypersetup{
  colorlinks   = true, %Colours links instead of ugly boxes
  urlcolor     = black, %Colour for external hyperlinks
  linkcolor    = black, %Colour of internal links
  citecolor   = black %Colour of citations
}
\usepackage[activate={true,nocompatibility},final,tracking=true,kerning=true,spacing=true,factor=1100,stretch=10,shrink=10]{microtype}
\frenchspacing
\usepackage[nodayofweek,level]{datetime}
\usepackage{calc,url}
\newcounter{qz}\setcounter{qz}{0}
\newcommand{\qz}{%\
\setcounter{qz}{\value{qz}+1}
\textbf{In-class  \theqz} \,}

\newcounter{hw}\setcounter{hw}{0}
\newcommand{\hw}{%\
\setcounter{hw}{\value{hw}+1}
\textbf{HW \thehw}}

\newcounter{ex}\setcounter{ex}{0}
\newcommand{\ex}{%\
\setcounter{ex}{\value{ex}+1}
Exam \theex}

\usepackage[T1]{fontenc} 
\usepackage{fourier}
%\usepackage{tgschola} %to look retro
\newenvironment{mypar}[2]
  {\begin{list}{}%
    {\setlength\leftmargin{#1}
    \setlength\rightmargin{#2}}
    \item[]}
  {\end{list}}


\newcounter{wk}\setcounter{wk}{0}
\newcommand{\wk}{%\
\setcounter{wk}{\value{wk}+1}
\thewk \,\,}

\usepackage[nomessages]{fp}% http://ctan.org/pkg/fp


\usepackage{enumerate}
\usepackage{graphicx}

\usepackage{paralist}
\renewenvironment{description}[0]{\begin{compactdesc}}{\end{compactdesc}}

\newenvironment{alphalist}{
  \begin{enumerate}[(a)]
    \addtolength{\itemsep}{-0.75\itemsep}}
  {\end{enumerate}}
  \cleanlookdateon% Remove ordinal day reference
  \newcommand{\RomanNumeralCaps}[1]
      {\MakeUppercase{\romannumeral #1}}

\usepackage{xspace}
\makeatletter
\DeclareRobustCommand{\maybefakesc}[1]{%
  \ifnum\pdfstrcmp{\f@series}{\bfdefault}=\z@
    {\fontsize{\dimexpr0.8\dimexpr\f@size pt\relax}{0}\selectfont\uppercase{#1}}%
  \else
    \textsc{#1}%
  \fi
}
\newcommand\AM{\,\maybefakesc{am}\xspace}
\newcommand\PM{\,\maybefakesc{pm}\xspace}
\makeatother

 \newcommand{\coursename}{Foundations of Mathematics}
\newcommand{\coursenumber}{MATH 250}
\newcommand{\sectionnumber}{01}
\newcommand{\term}{Spring }
\newcommand{\room}{Discovery Hall, room 386}
\newcommand{\meetingtime}{This class meets Monday, Wednesday, and Friday in \room \/  from 2:30 \PM to 3:20:PM.}
\newcommand{\officehours}{Either in person or by Zoom: Monday, Wednesday, and Friday 9:30\AM -- 11:00\AM,
    Tuesday and Thursday 1:00\PM -- 2:00\PM, and by appointment.}

\begin{document}
\cleanlookdateon% Remove ordinal day reference
\shortdate
\printyearoff
\large
\begin{center}
    \textbf{\coursename}  \\
    {\coursenumber--\sectionnumber} \\
     {\term \the\year} \\
\end{center}

\vskip0.25in
\normalsize


\begin{center}
\begin{description}
    \item[Instructor:] Dr.\  Willis, Professor of Mathematics
    \item[Office:]  Discovery Hall, Room 368
    \item[\phone:]   \phonenumber[country=US]{3088658868}
    \item[\Email:]    \href{mailto:willisb@unk.edu}{willisb@unk.edu}
    \item[Zoom for classes:] For Zoom class meetings, use the Meeting ID: 616 568 5706. 
    \item[Office Hours:] \officehours
  \end{description}
\end{center}

\subsubsection*{Class meeting time and place}

\meetingtime



\subsubsection*{Course Resources}

\noindent Our textbook is \emph{Introduction to Mathematical Structures and Proofs}, \nth{2} edition,  by  Larry Gerstein.
Some homework assignments for this course will need to be typeset. To do this, you will need to create a \emph{no cost} 
account on Overleaf (\url{https://www.overleaf.com/}).   For  tutorial for using Overleaf, see \url{https://www.overleaf.com/tutorial}.



\subsubsection*{Important Dates}

\begin{mypar}{0.25in}{0.25in} 

      \textbf{First Homework due} \dotfill  \printdate{28/1/\the\year}  \\
       \textbf{Exam 1} \dotfill \printdate{17/2/\the\year}  \\
    \textbf{Exam 2} \dotfill  \printdate{2/3/\the\year} \\
    \textbf{Exam 3} \dotfill \printdate{14/4/\the\year} \\
      \textbf{Final exam} \dotfill  \printdate{15/5/\the\year} 3:30 \PM  --  5:30 \PM
\end{mypar}



\subsubsection*{Grading}

Your course grade will be based on weekly homework sets, three midterm exams, and a comprehensive 
final exam; specifically:
\begin{mypar}{0.25in}{0.25in}
    \textbf{Weekly Homework:}  \emph{12 ten point assignments}  \dotfill 120 (total) \\
    \textbf{Mid-term exams 1,2, and 3:} \emph{100 points each} \dotfill 300 (total)\\
      \textbf{Comprehensive Final exam} \dotfill 150 (total)
\end{mypar}
If it is necessary to adjust the number of  homework assignments,  your homework point 
total will be scaled to a total of 120.  For example, if we have only eleven homework sets, your homework score will
be scaled by a factor of \(120/110\).

\FPeval{\points}{round(120+300+150,0)}

\FPeval{\F}{round(\points*0.6-1,0)}
\FPeval{\Dm}{round(\points*0.6,0)}
\FPeval{\D}{round(\points*0.63,0)}
\FPeval{\Dp}{round(\points*0.66,0)}

\FPeval{\Cm}{round(\points*0.7,0)}
\FPeval{\C}{round(\points*0.73,0)}
\FPeval{\Cp}{round(\points*0.76,0)}

\FPeval{\Bm}{round(\points*0.8,0)}
\FPeval{\B}{round(\points*0.83,0)}
\FPeval{\Bp}{round(\points*0.86,0)}

\FPeval{\Am}{round(\points*0.9,0)}
\FPeval{\A}{round(\points*0.93,0)}
\FPeval{\Ap}{round(\points*0.97,0)}
The following table shows the \emph{minimum} number of points (out of \points) that
are required for each of the twelve letter grades D- through A+. For
example, a point total of \Bp\/  points will earn you a grade of B+,  and 
a point total of \Am\/ points will earn you a grade of A-. A point
total of \F\/  or less earns you a failing course grade.
 
 \vspace{0.1in}
     \begin{minipage}{5.5in}
  \centering 
\begin{mypar}{0.25in}{0.25in}
    \begin{minipage}{2.5in}
        D-  \dotfill \Dm \\
        D \dotfill \D \\
        D+ \dotfill \Dp \\
        C- \dotfill \Cm  \\
        C \dotfill \C \\
        C+ \dotfill \Cp 
        \end{minipage}
    \phantom{xxx}
    \begin{minipage}{2.5in}
        B- \dotfill \Bm \\
        B \dotfill  \B \\
        B+ \dotfill  \Bp\\
        A- \dotfill  \Am \\
        A \dotfill  \A \\
        A+ \dotfill  \Ap
    \end{minipage}
\end{mypar} 
\end{minipage}

\subsubsection*{Prerequisite}

The prerequisite for MATH 250 is an earned grade of D- or higher in either MATH 115 or MATH 123.

\subsubsection*{Catalog description}

\textbf{Foundations of Mathematics  (3 credit hours)} Topics of sets and symbolic logic are studied with the objective of using them in the detailed study of the nature of different types of proofs used in mathematics. Also, the processes of problem solving are studied for developing strategies of problem solving.

\subsubsection*{Learning Outcomes}

On completion of this course, students will
\begin{alphalist}
    \item gain an understanding of na\"ive set theory. 
    \item gain an understanding of symbolic logic, quantifiers, and functions.
    \item gain an understanding of direct proofs, proofs by contradiction, proofs by contrapositive, and proofs by induction.
    \item gain the ability to read and understand mathematical proofs.
    \item gain the problem solving skills that are needed to create a mathematical proof.
\end{alphalist}

\subsubsection*{Course Calendar}

Generally, we'll adhere to the scheduled exam dates even if we are ahead or behind with course work.  
When we are ahead or behind, the topics on the exams will be appropriately adjusted.  


\vspace{0.1in}
\noindent \textbf{Notices:}


\begin{alphalist}
   \item \emph{Exams will be given on the \textbf{Friday} of the week they are assigned.}
   

    \item Homework (\textbf{HW}) will be due at midnight on  Saturday of the week they are assigned.  


\end{alphalist}

\vspace{0.1in}

\begin{center}
    \small
\begin{tabular}  {|l|l|l|l|l|}
\hline
{\bf Week}  & \textbf{Week Starting} &  {\bf Section(s)} & {\bf Topic(s)} & \textbf{Assessment} \\
\hline \hline 
\wk    &  \printdate{23/1/\the\year} &    \S 1.1, \S 1.2  &  Logical connectives; Truth tables  & \hw  \\
\wk    & \printdate{30/1/\the\year}   &  \S1.3 &  Introduction to Overleaf, Conditional statements & \hw  \\
\wk    & \printdate{6/2/\the\year}&     \S1.4, \S1.5  &   Proof structures;  Logical equivalence &  \hw \\
\wk    & \printdate{13/2/\the\year}   &     \S2.1, \S2.2  & Sets; Russell's paradox   &   \textbf{\ex}       \\ \hline
\wk    & \printdate{21/2/\the\year} &  \S2.3, \S2.4    &  Quantifiers; Set inclusion   & \hw \\ 
\wk    & \printdate{20/2/\the\year}    & \S2.5, \S2.6   &  Union, intersection, complements; Indexed sets &    \hw  \\
\wk    & \printdate{27/2/\the\year}     & \S2.7, \S2.8  &  Power sets; Ordered pairs & \hw \\
\wk    & \printdate{6/3/\the\year}   & \S2.9  &  Set decompositions     &   \textbf{\ex}   \\ \hline
\wk   &  \printdate{20/3/\the\year}   & \S2.10 &  Induction   & \hw \\ 
\wk   &  \printdate{27/3/\the\year}      &   \S3.1 &  Functions &   \hw \\
\wk   &  \printdate{3/4/\the\year}   &   \S3.2, \S3.3 & One-to-one and onto functions; Function composition   & \hw  \\
\wk   & \printdate{10/4/\the\year}  & \S4.1, \S4.2  &  Cardinality; Finite and infinite sets   &   \textbf{\ex}  \\ \hline
\wk   & \printdate{17/4/\the\year} & \S4.3  &   Countable  and uncountable sets & \hw \\
\wk   & \printdate{24/4/\the\year}    &  \S5.1, \S5.2      & Combinatorial problems; Addition and product rules  &  \hw   \\
\wk   & \printdate{1/5/\the\year}   &  \S5.3      & Permutations    &     \hw   \\
\wk   & \printdate{8/5/\the\year}   &  \S5.4      & Permutations and geometric symmetry   &       \\ \hline
 \wk   & \printdate{15/5/\the\year}     &  &    \hfill  & \textbf{ Final Exam}  \\  \hline
   
\end{tabular}
\end{center}



\subsubsection* {Policies}

Unless an assessment is \emph{explicitly} stated to be a group project,  \emph{all work you turn in for a grade must be your own.}  If you need assistance in completing a homework assignment, you may ask me for help. Googling for answers, seeking help from the Learning Commons or other faculty members,  or using solution keys from previous terms (either from UNK or other universities) is also prohibited.  Violation of these rules will result in earning a grade of zero on the assessment. Each homework assignment you turn in for a grade must include the statement:

\begin{quote}
\fbox{I have neither given nor received unauthorized assistance on this assignment.}
\end{quote}
 If two assignments are so similar that only collaboration could explain their similarities, both assignments will receive a grade of zero.  Using unauthorized materials or communication devices (cell phone, for
example) while taking a test will earn you a grade of zero on that assessment.  
For the university academic integrity policy, please read the footnote.{\small  \url{https://catalog.unk.edu/undergraduate/academics/academic-regulations/academic-integrity-policy/}
 

\begin{enumerate}

\item Regular in person class attendance is required. If you are ill or need to miss 
class due to athletics, please let me know ahead of time and I will make an effort to put the class on Zoom. 
Our classroom technology often doesn't work, so do not rely on watching recorded classes.

\item There is no explicit grade penalty for not attending class. But if you choose to not attend class for reasons other
than illness or athletics, I reserve the right to not be all that helpful in giving you assistance on homework or helping 
you learn missed material.

\item All examinations, including the final exam, must be taken in person.

\item For examinations and in class assignments, show your work.  \emph{No credit will be given for multi-step problems without the necessary work. Your solution must contain enough detail
so that I am convinced that you could correctly work any similar problem.} Also erase or clearly mark any work you want me to ignore; otherwise,
I'll grade it.  

\item The work you turn in is expected to be \emph{accurate, 
complete, concise, neat}, and \emph{well-organized}.  
\emph{You will not earn full credit on work that falls short of 
these expectations.}

\item Class cancellations due to weather, illness, or other 
unplanned circumstances may require that we make  adjustments
to the course calendar, exam dates, and due dates or specifics for 
course assessments. 


\item Extra credit is not allowed. 



\item For examinations, you may use a teacher provided quick reference sheet, 
but no other reference materials. You may also use a pencil, eraser, 
and a scientific calculator. For examinations, your phone and all such
devices must be turned off and \emph{out of sight}. 

\item Generally, if you are ill or absent for any reason (including 
athletics), you must turn in your in class work on time. Permission to
turn in work late must be made before the due date, otherwise late in class work 
will count zero points.


 

\item During class time, please refrain from using electronic devices. If your 
device usage distracts your classmates, I will ask you to put it away. If it's my 
impression that you are often not paying attention in class, I reserve the right to 
decline to help you during office hours.

\item The final examination will be \emph{comprehensive} and it will be given 
during the  time scheduled by the University. Except for \emph{extraordinary circumstances}
you must take the exam at this time.
 
\item If you have questions about how your work has been graded, make an appointment with me immediately.


\item Please regularly check Canvas  to verify that your scores have 
been recorded correctly.  If I made a mistake in recording one of
your grades, I'll correct it provided you saved your paper.

\end{enumerate}

\newpage

\subsubsection*{Reporting Student Sexual Harassment, Sexual Violence or Sexual Assault}

Reporting allegations of rape, domestic violence, dating violence, sexual assault, sexual harassment, and stalking enables the University to promptly provide support to the impacted student(s), and to take appropriate action to prevent a recurrence of such sexual misconduct and protect the campus community. Confidentiality will be respected to the greatest degree possible. Any student who believes they may be the victim of sexual misconduct is encouraged to report to one or more of the following resources:
\begin{description}
    \item[Local Domestic Violence, Sexual Assault Advocacy Agency]\phonenumber[country=US]{3082372599}

    \item[Campus Police (or Security)]\phonenumber[country=US]{3088658911}
    
    \item[Title IX Coordinator]\phonenumber[country=US]{3088658655}

\end{description}
Retaliation against the student making the report, whether by students or University employees, will not be tolerated.

\subsubsection*{Students with Disabilities}

It is the policy of the University of Nebraska at Kearney to provide flexible and individualized reasonable accommodation to students with documented disabilities. To receive accommodation services for a disability, students must be registered with the UNK Disabilities Services for Students (DSS) office, 175 Memorial Student Affairs Building,  \phonenumber[country=US]{3088658214} or by email unkdso@unk.edu  


\subsubsection*{Students Who are Pregnant}

It is the policy of the University of Nebraska at Kearney to provide flexible and individualized reasonable accommodation to students who are pregnant. To receive accommodation services due to pregnancy, students must contact the Student Health office at \phonenumber[country=US]{3088658218}. The following links provide information for students and faculty regarding pregnancy rights\footnote{\small  \url{https://thepregnantscholar.org/title-ix-basics/}, \url{https://nwlc.org/resource/faq-pregnant-and-parenting-college-graduate-students-rights/UNK Statement of Diversity & Inclusion}}

\subsubsection*{UNK Statement of Diversity \& Inclusion}

UNK stands in solidarity and unity with our students of color, our Latinx and international students, our LGBTQIA+ students and students from other marginalized groups in opposition to racism and prejudice in any form, wherever it may exist. It is the job of institutions of higher education, indeed their duty, to provide a haven for the safe and meaningful exchange of ideas and to support peaceful disagreement and discussion. In our classes, we strive to maintain a positive learning environment based upon open communication and mutual respect. UNK does not discriminate on the basis of race, color, national origin, age, religion, sex, gender, sexual orientation, disability or political affiliation. Respect for the diversity of our backgrounds and varied life experiences is essential to learning from our similarities as well as our differences. The following link provides resources and other information regarding D\&I:  \url{https://www.unk.edu/about/equity-access-diversity.php}


\end{document}
For the UNK's Policies and statements on: Attendance Policy, Academic Honesty Policy, 
Reporting Student Sexual Harassment, Sexual Violence or Sexual Assault, Students 
with Disabilities, Students Who are Pregnant, and UNK Statement of Diversity 
\& Inclusion,  you \emph{must read}
\url{https://www.unk.edu/academic_affairs/asa_forms/course-policies-and-resources.php}.
\end{enumerate}
%\includepdf[pages={1-},angle=90]{door_schedule.pdf}  
%\includepdf[pages={1-},angle=90]{analysis-quick-reference.pdf} 
\end{document}
\subsubsection* {Policies}

Unless an assessment is \emph{explicitly} stated to be a group project,  \emph{all work you turn in for a grade must be your own.}  If you need assistance in completing a homework assignment, you may ask me for help. Googling for answers, seeking help from the Learning Commons or other faculty members,  or using solution keys from previous terms (either from UNK or other universities) is also prohibited.  Violation of these rules will result in earning a grade of zero on the assessment. Further each homework assignment you turn in for a grade must include the statement:

\begin{quote}
\fbox{I have neither given nor received unauthorized assistance on this assignment.}
\end{quote}
 If two assignments are so similar that only collaboration could explain their similarities, both assignments will receive a grade of zero.  Using unauthorized materials or communication devices (cell phone, for
example) while taking a test will earn you a \emph{failing course grade.}  


Examinations are closed book and closed notes; using
unauthorized materials while taking a test will earn you a failing
course grade.  

\begin{enumerate}

\item For examinations and in class assignments, show your work.  \emph{No credit will be given for multi-step problems without the necessary work. Your solution must contain enough detail
so that I am convinced that you could correctly work any similar problem.} Also erase or clearly mark any work you want me to ignore; otherwise,
I'll grade it.  

\item The work you turn in is expected to be \emph{accurate, 
complete, concise, neat}, and \emph{well-organized}.  
\emph{You will not earn full credit on work that falls short of 
these expectations.}

\item Class cancellations due to weather or illness or other 
unplanned circumstances may require that we make  adjustments
to the course calendar, exam dates, and due dates or specifics for 
course assessments. 


\item Extra credit is not allowed. 

%\item For online homework and in class work, you may work in groups and you may 
%seek help from the Learning Commons. 

\item For examinations, you make use a teacher provided crib sheet, 
but no other reference materials. You may also use a pencil, eraser, 
and a scientific calculator. For examinations, your phone and all such
devices must be turned off and \emph{out of sight}. Checking your phone
to look at the time is \emph{not} allowed. Using unauthorized 
materials during an examination will earn you a failing course grade.

\item Generally, if you are ill or absent for any reason (including 
athletics), you must turn in your in class work on time. Permission to
turn in work late must be made in advance, otherwise late in class work 
will count zero points.


 

\item During class time, please refrain from using electronic devices. If your 
device usage distracts your classmates, I will ask you to put it away. If it's my 
impression that you are often not paying attention in class, I reserve the right to 
decline to help you during office hours.

\item The final examination will be \emph{comprehensive} and it will be given 
during the 
time scheduled by the University. Except for \emph{extraordinary circumstances}
you must take the exam at this time.


 
\item If you have questions about how your work has been graded, make an appointment with me immediately.

\item All printed materials, in either paper or digital form, that I 
provide for you in this class, are for your own use. Re-posting or 
sharing these materials with other persons is prohibited. 

\item Please regularly check Canvas  to verify that your scores have 
been recorded correctly.  If I made a mistake in recording one of
your grades, I'll correct it provided you saved your paper.

\subsubsection*{Course Objectives} 

\noindent Students will learn to read and to construct \emph{direct proofs}, \emph{proofs by contradiction}, and \emph{inductive proofs}.  Students will also learn the fundamentals of \emph{boolean logic, sets, and functions}.



\subsubsection*{Prerequisite}

The prerequisite for MATH 250 is a grade of D- or higher in either MATH 115 or MATH 123.

\subsubsection*{Students with Disabilities or Those Who are Pregnant}

\paragraph{Students with Disabilities} It is the policy of the University of Nebraska 
at Kearney to provide flexible and individualized reasonable 
accommodation to students with documented disabilities. To receive 
accommodation services for a disability, students must be
registered with the UNK Disabilities Services for Students (DSS) 
office, 175 Memorial Student Affairs Building, 
\phonenumber[country=US]{3088658214} or 
by email \href{mailto:unkdso@unk.edu}{unkdso@unk.edu}

\paragraph{UNK Statement of Diversity \& Inclusion:} UNK stands in solidarity and unity with our students of color, our Latinx and international students, our LGBTQIA+ students and students from other marginalized groups in opposition to racism and prejudice in any form, wherever it may exist. It is the job of institutions of higher education, indeed their duty, to provide a haven for the safe and meaningful exchange of ideas and to support peaceful disagreement and discussion. In our classes, we strive to maintain a positive learning environment based upon open communication and mutual respect. UNK does not discriminate on the basis of race, color, national origin, age, religion, sex, gender, sexual orientation, disability or political affiliation. Respect for the diversity of our backgrounds and varied 
life experiences is essential to learning from our similarities as well as our differences. The following link provides resources and other information regarding D\&I: 
\url{https://www.unk.edu/about/equity-access-diversity.php}

\paragraph{Students Who are Pregnant} It is the policy of the University of Nebraska at Kearney to provide flexible and individualized reasonable accommodation to students who are pregnant. To receive accommodation services due to pregnancy, students must contact Cindy Ference in Student Health, 308-865-8219. The following link provides information for students and faculty regarding pregnancy rights.\footnote{\small  \url{http://www.nwlc.org/resource/pregnant-and-parenting-students-rights-faqs-college-and-graduate-students}}

\paragraph{Reporting Student Sexual Harassment, Sexual Violence or Sexual Assault} Reporting allegations of rape, domestic violence, dating violence, sexual assault, sexual harassment, and stalking enables the University to promptly provide support to the impacted student(s), and to take appropriate action to prevent a recurrence of such sexual misconduct and protect the campus community. Confidentiality will be respected to the greatest degree possible. Any student who believes she or he may be the victim of sexual misconduct is encouraged to report to one or more of the following resources:
\begin{alphalist}
\item Local Domestic Violence, Sexual Assault Advocacy Agency \phonenumber[country=US]{3082372599}

\item Campus Police (or Security) \phonenumber[country=US]{3088658911}

\item Title \RomanNumeralCaps{9} Coordinator \phonenumber[country=US]{3088658655}

\end{alphalist}
Retaliation against the student making the report, whether by students or University employees, will not be tolerated.If you have questions regarding the information in this email please 
contact Mary Chinnock Petroski, Chief Compliance Officer (
   \href{mailto:petroskimj@unk.edu}{petroskimj@unk.edu} 
    or phone \phonenumber[country=US]{3088658400}.

\subsubsection*{University policy on wearing face masks}    
    All vaccinated faculty, staff and students, as well 
    as visitors to campus, are encouraged, but not required, 
    to wear face masks indoors. Unvaccinated individuals 
    should continue to wear masks. (Masks are required indoors for all 
    individuals at the University of Nebraska Medical Center.\footnote{\small
    \url{https://nebraska.edu/news-and-events/news/2021/08/university-of-nebraska-system-updates-covid-19-protocols}}

 

    Consistent with the University of Nebraska’s guiding protocols, there 
    are circumstances in which many people in our community may not be able to 
    be vaccinated. Faculty may request, not require, that students wear masks in 
    class under circumstances that clearly indicate unnecessary yet controllable 
    risk of infection.  Please note that pandemic precautions are subject to change.




\end{enumerate}



\end{document}

%------------
\newpage

\subsubsection*{Course Calendar}


Our goal is to cover \S1.1--\S6.2. We will try to adhere to the following schedule,
but we will modify it if necessary.  Homework is due on the last class day of the week they are scheduled.


Regardless of how much material we have covered, \emph{the examinations will almost certainly be given
on the indicated days.}

\vspace{0.1in}

\begin{tabular} {|r| l | l | l |} 
\hline
Week & Week of &  Section(s) & Topic or Exam \\ \hline \hline
\wk    &  \formatdate{\value{cd}}{\value{cmon}}{\value{cd}} &   \S1.1--1.2 & Logic \hfill  Homework \hw  \\
\setcounter{cd}{\value{cd}+7}

\wk    &  \formatdate{\value{cd}}{\value{cmon}}{\value{cy}} &   \S1.3--1.4& Logic \hfill  Homework \hw   \\ 
\setcounter{cd}{\value{cd}+7} \setcounter{cd}{3}  \setcounter{cmon}{\value{cmon}+1}

\wk    &  \formatdate{\value{cd}}{\value{cmon}}{\value{cy}} &    \S1.4--1.5  & Logic \hfill  Homework \hw   \\
\setcounter{cd}{\value{cd}+7} 

\wk    &  \formatdate{\value{cd}}{\value{cmon}}{\value{cy}}&    \S2.1--2.2  & Sets \hfill  Homework \hw   \\
\setcounter{cd}{\value{cd}+7} 

\wk    &  \formatdate{\value{cd}}{\value{cmon}}{\value{cy}}&   \S2.3--2.5 & Sets  \hfill    \textbf{Exam} \ex, Friday 21 September \\ \hline 
\setcounter{cd}{\value{cd}+7} 

\wk    &  \formatdate{\value{cd}}{\value{cmon}}{\value{cy}} &   \S2.6--2.8 & Sets \hfill  Homework \hw   \\
\setcounter{cmon}{\value{cmon}+1} \setcounter{cd}{1}


\wk    & \formatdate{\value{cd}}{\value{cmon}}{\value{cy}}&        \S2.9--2.10  & Sets, Recursion, and Induction  \hfill  Homework \hw \\
\setcounter{cd}{\value{cd}+7}

\wk    &  \formatdate{\value{cd}}{\value{cmon}}{\value{cy}} &     \S3.1--3.2 & Functions \hfill  Homework \hw   \\
\setcounter{cd}{\value{cd}+7}


\wk    &  \formatdate{\value{cd}}{\value{cmon}}{\value{cy}}&     \S3.3 & Functions  \hfill  Homework \hw   \\ 
\setcounter{cd}{\value{cd}+7}

\wk  & \formatdate{\value{cd}}{\value{cmon}}{\value{cy}} &      \S4.1--4.2 &  Finite and infinite sets \hfill \textbf{Exam} \ex, Friday 26 October    \\ \hline 
\setcounter{cd}{\value{cd}+7}

\wk   & \formatdate{\value{cd}}{\value{cmon}}{\value{cy}} &      \S4.3--4.4 &   Finite and infinite sets  \hfill  Homework \hw  \\ 
\setcounter{cmon}{\value{cmon}+1} \setcounter{cd}{5}

\wk  & \formatdate{\value{cd}}{\value{cmon}}{\value{cy}}&      \S5.1--5.2 &  Combinatorics \hfill  Homework \hw  \\
\setcounter{cd}{\value{cd}+7}

\wk  & \formatdate{\value{cd}}{\value{cmon}}{\value{cy}}&      \S5.3--5.5 &  Combinatorics \hfill  Homework \hw   \\
\setcounter{cd}{\value{cd}+7}

\wk   & \formatdate{\value{cd}}{\value{cmon}}{\value{cy}} &      \S5.6--5.8 &   Combinatorics \hfill  Homework \hw  \\ 
\setcounter{cd}{\value{cd}+7}

\wk   & \formatdate{\value{cd}}{\value{cmon}}{\value{cy}}&      \S6.1  &  Number Theory    \hfill \textbf{Exam} \ex, Friday 30 November   \\
\setcounter{cmon}{\value{cmon}+1} \setcounter{cd}{3}

\wk   & \formatdate{\value{cd}}{\value{cmon}}{\value{cy}}&      \S6.2  &  Number Theory   \\
\setcounter{cd}{\value{cd}+7}

\wk   & \formatdate{\value{cd}}{\value{cmon}}{\value{cy}}  &    & \hfill \textbf{Final Exam},  Wednesday  12 December,   13:00--15:00  \\ \hline
\end{tabular}
  
\vfill

%\copyright  \, 2010 Barton Willis

\end{document}

